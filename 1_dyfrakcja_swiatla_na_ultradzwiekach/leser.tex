\documentclass[a4paper,12pt]{article}

\usepackage[T1]{fontenc}
\usepackage[utf8]{inputenc}
\usepackage[english, polish]{babel}
\usepackage{lmodern}
\usepackage{graphicx}
\usepackage{fancyhdr}
\usepackage{float}
\usepackage{array}

%\usepackage{mathtools}


\setlength{\textheight}{23.5cm}
\setlength{\textwidth}{15.92cm}
\setlength{\footskip}{10mm}
\setlength{\oddsidemargin}{0mm}
\setlength{\evensidemargin}{0mm}
\setlength{\topmargin}{0mm}
\setlength{\headsep}{15mm}
\setlength{\parindent}{0cm}
\setlength{\parskip}{2.5mm}
%nowa extra row do tabeli :)  :) 
\setlength{\extrarowheight}{4pt}

\author{Justyna Ilczuk, Jacek Rosiński}

\begin{document}

\begin{center}

    \begin{tabular}{ | m{5cm}| m{5cm} | m{5cm} |}
    \hline 
    \multicolumn{2}{|c|}{{ \Large \textbf{Laboratorium Fizyki 2}} }
    &  
    \begin{center}
    Data wykonania ćwiczenia:
    \end{center}
    \begin{center}
      9.10.2013 
    \end{center}
    \begin{center}
    Środa 9.45-12.45
    \end{center}
     \\ 
    
    \hline
    \multicolumn{2}{|c|}{Justyna Ilczuk \newline Jacek Rosiński}
    & \begin{center}
    {\small Data złożenia sprawozdania:} \newline \today
    \end{center}   \\
   	
   	\hline
    Wydział Fizyki & Grupa: K-1 \newline Rok akademicki: 2013/2014 &    Nr ćwiczenia: 7 \\
   	\hline
   	\multicolumn{2}{|l|}{Prowadzący: Jan Grabski i ktoś jeszcze?} & \multicolumn{1}{|l|}{Ocena końcowa:}\\
    \hline
    \end{tabular}
\end{center}

%\newpage



\section{Cel ćwiczenia}

Celem ćwiczenia było zbadanie zjawisk jakie towarzyszą przejściu fali świetlnej przez ośrodek o zmiennym współczynniku załamania wywołanym falami ultradźwiękowymi. Porównanie wyników otrzymanych różnymi metodami i przeprowadzenie dyskusji nad samym zjawiskiem było również ważną częścią doświadczenia. 

\newpage

\pagestyle{fancy}
\fancyfoot[CO]{\ }
\fancyhead[RO]{\footnotesize{\thepage} }
%\fancyhead[RO]{\footnotesize{\ } }
\fancyhead[LO]{Justyna Ilczuk i Jacek Rosiński K-1, Dyfrakcja fali świetlnej na fali ultradźwiękowej }

\section{Wstęp}

% wrzucanie wykresów:

%\begin{figure} [H]
%  \begin{center}
%    \includegraphics[width = 10cm]{../Obrazki_i_tekst/obrobione/u3.png}
%    \caption{Układ 3}
%  \end{center}
%\end{figure}

W eksperymencie mogliśmy mieć do czynienia z dwoma rodzajami dyfrakcji:
\begin{itemize}
\item dyfrakcją Ramana-Natha
\item dyfrakcją braggowską
\end{itemize}

Rodzaj dyfrakcji można określić na podstawie parametru Kleina-Cooke'a, który wyraża się następującym wzorem

\(
Q = \frac{K^2  L}{k cos \alpha }
\)
\\
Gdzie
\(
K = \frac{2 \pi}{ \lambda }; \\ k = \frac{2 \pi}{\lambda}
\) wektory falowe fal: ultradźwiękowej i świetlnej

\( L  \) - długość wnęki

Jeżeli \( Q << 1\) to mamy do czynienia z dyfrakcją Ramana-Nata

Jeżeli \( Q >> 1 \) to mamy do czynienia z dyfrakcją braggowską

\section{Użyty sprzęt i układy pomiarowe}

% tutaj trzeba wrzucić schemat układu pomiarowego i opis użytych przyrządów
% a także szczegóły techniczne, specyficzne dla naszego setupu eksperymentu



\section{Opracowanie wyników}
Co trzeba opracować?
\begin{itemize}
\item wyznaczyć długości fali na podstawie wzoru dla siatki dyfrakcyjnej i jednocześnie wyznaczyć tę wartość za pomocą zmiany długości wnęki rezonansowej dla fali ultradźwiękowych.

\item Jak dokładnie to wyszło? Czy wyniki z obu rodzajów pomiarów, są podobne, etc.

\item znaleźć krzywą dyspersji (prędkość fazowa fali ultradźwiękowej w zależności od częstotliwości (porównać z tablicami)

\item parametr Kleina-Cooke'a - jaki rodzaj dyfrakcji występował w eksperymencie
\end{itemize}




\section{Wnioski}

Nie ma idealnych wyników ani idealnych teorii.

\begin{itemize}
  \item ważne 1
  \item ważne 2
  
\end{itemize}

\end{document}
