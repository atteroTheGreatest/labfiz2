\documentclass[a4paper,11pt]{article}

\usepackage[T1]{fontenc}
\usepackage[utf8]{inputenc}
\usepackage[english, polish]{babel}
\usepackage{lmodern}
\usepackage{graphicx}
\usepackage{fancyhdr}
\usepackage{float}
\usepackage{array}

%\usepackage{mathtools}


\setlength{\textheight}{23.5cm}
\setlength{\textwidth}{15.92cm}
\setlength{\footskip}{10mm}
\setlength{\oddsidemargin}{0mm}
\setlength{\evensidemargin}{0mm}
\setlength{\topmargin}{0mm}
\setlength{\headsep}{15mm}
\setlength{\parindent}{0cm}
\setlength{\parskip}{2.5mm}
\author{Justyna Ilczuk, Jacek Rosiński}

\begin{document}

\begin{center}

    \begin{tabular}{ | m{5cm}| m{5cm} | m{5cm} |}
    \hline 
    \multicolumn{2}{|c|}{Labfiz2}
    & Rok akademicki 2013-2014 \\ 
    
    \hline
    Środa 9.45-12.45
    & Justyna Ilczuk \newline Jacek Rosiński
    & Wykonane w dniu xx.xx.2013 \\
   	
   	\hline
   	Ćwiczenie 6 & Nazwa ćwiczenia &    Ocena: \\
   	\hline
    \end{tabular}
\end{center}

%\newpage
\pagestyle{fancy}
\fancyfoot[CO]{\ }
\fancyhead[RO]{\footnotesize{\thepage} }
%\fancyhead[RO]{\footnotesize{\ } }
\fancyhead[LO]{Justyna Ilczuk i Jacek Rosiński K-1, nazwa ćwiczenia }


\section{Cel ćwiczenia}

\section{Użyty sprzęt i układy pomiarowe}


\end{itemize}

\section{Wstęp teoretyczny}

% wrzucanie wykresów:

%\begin{figure} [H]
%  \begin{center}
%    \includegraphics[width = 10cm]{../Obrazki_i_tekst/obrobione/u3.png}
%    \caption{Układ 3}
%  \end{center}
%\end{figure}


\section{Opracowanie wyników}


\subsection{Subsekcja}


\section{Wnioski}

Nie ma idealnych wyników ani idealnych teorii.

\begin{itemize}
  \item ważne 1
  \item ważne 2
  
\end{itemize}

\end{document}